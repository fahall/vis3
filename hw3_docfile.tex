%%%%%%%%%%%%%%%%%%%%%%%%%%%%%%%%%%%%%%%%%
% Structured General Purpose Assignment
% LaTeX Template
%
% This template has been downloaded from:
% http://www.latextemplates.com
%
% Original author:
% Ted Pavlic (http://www.tedpavlic.com)
%
% Note:
% The \lipsum[#] commands throughout this template generate dummy text
% to fill the template out. These commands should all be removed when 
% writing assignment content.
%
%%%%%%%%%%%%%%%%%%%%%%%%%%%%%%%%%%%%%%%%%

%----------------------------------------------------------------------------------------
%	PACKAGES AND OTHER DOCUMENT CONFIGURATIONS
%----------------------------------------------------------------------------------------

\documentclass{article}

\usepackage{fancyhdr} % Required for custom headers
\usepackage{lastpage} % Required to determine the last page for the footer
\usepackage{extramarks} % Required for headers and footers
\usepackage{graphicx} % Required to insert images
\usepackage{lipsum} % Used for inserting dummy 'Lorem ipsum' text into the template

% Margins
\topmargin=-0.45in
\evensidemargin=0in
\oddsidemargin=0in
\textwidth=6.5in
\textheight=9.0in
\headsep=0.25in 

\linespread{1.1} % Line spacing

% Set up the header and footer
\pagestyle{fancy}
\lhead{\hmwkAuthorName} % Top left header
\chead{\hmwkClass\ (\hmwkClassInstructor\ \hmwkClassTime): \hmwkTitle} % Top center header
\rhead{\firstxmark} % Top right header
\lfoot{\lastxmark} % Bottom left footer
\cfoot{} % Bottom center footer
\rfoot{Page\ \thepage\ of\ \pageref{LastPage}} % Bottom right footer
\renewcommand\headrulewidth{0.4pt} % Size of the header rule
\renewcommand\footrulewidth{0.4pt} % Size of the footer rule

\setlength\parindent{0pt} % Removes all indentation from paragraphs

%----------------------------------------------------------------------------------------
%	DOCUMENT STRUCTURE COMMANDS
%	Skip this unless you know what you're doing
%----------------------------------------------------------------------------------------

% Header and footer for when a page split occurs within a problem environment
\newcommand{\enterProblemHeader}[1]{
\nobreak\extramarks{#1}{#1 continued on next page\ldots}\nobreak
\nobreak\extramarks{#1 (continued)}{#1 continued on next page\ldots}\nobreak
}

% Header and footer for when a page split occurs between problem environments
\newcommand{\exitProblemHeader}[1]{
\nobreak\extramarks{#1 (continued)}{#1 continued on next page\ldots}\nobreak
\nobreak\extramarks{#1}{}\nobreak
}

\setcounter{secnumdepth}{0} % Removes default section numbers
\newcounter{homeworkProblemCounter} % Creates a counter to keep track of the number of problems

\newcommand{\homeworkProblemName}{}
\newenvironment{homeworkProblem}[1][Problem \arabic{homeworkProblemCounter}]{ % Makes a new environment called homeworkProblem which takes 1 argument (custom name) but the default is "Problem #"
\stepcounter{homeworkProblemCounter} % Increase counter for number of problems
\renewcommand{\homeworkProblemName}{#1} % Assign \homeworkProblemName the name of the problem
\section{\homeworkProblemName} % Make a section in the document with the custom problem count
\enterProblemHeader{\homeworkProblemName} % Header and footer within the environment
}{
\exitProblemHeader{\homeworkProblemName} % Header and footer after the environment
}

\newcommand{\problemAnswer}[1]{ % Defines the problem answer command with the content as the only argument
%\noindent\framebox[\columnwidth][c]{\begin{minipage}{0.98\columnwidth}#1\end{minipage}} % Makes the box around the problem answer and puts the content inside
{\bigskip
\begin{minipage}{0.98\columnwidth}#1\end{minipage}
\bigskip} 
}

\newcommand{\homeworkSectionName}{}
\newenvironment{homeworkSection}[1]{ % New environment for sections within homework problems, takes 1 argument - the name of the section
\renewcommand{\homeworkSectionName}{#1} % Assign \homeworkSectionName to the name of the section from the environment argument
\subsection{\homeworkSectionName} % Make a subsection with the custom name of the subsection
\enterProblemHeader{\homeworkProblemName\ [\homeworkSectionName]} % Header and footer within the environment
}{
\enterProblemHeader{\homeworkProblemName} % Header and footer after the environment
}
   
%----------------------------------------------------------------------------------------
%	NAME AND CLASS SECTION
%----------------------------------------------------------------------------------------

\newcommand{\hmwkTitle}{Homework\ \#3} % Assignment title
\newcommand{\hmwkDueDate}{Friday,\ April\ 3,\ 2015} % Due date
\newcommand{\hmwkClass}{CS\ 280} % Course/class
\newcommand{\hmwkClassTime}{9:30am} % Class/lecture time
\newcommand{\hmwkClassInstructor}{Malik} % Teacher/lecturer
\newcommand{\hmwkAuthorName}{Alex Hall and Rachel Albert} % Your name

\newcommand{\code}[1]{\newline\texttt{#1}\newline}

%----------------------------------------------------------------------------------------
%	TITLE PAGE
%----------------------------------------------------------------------------------------

\title{
\vspace{2in}
\textmd{\textbf{\hmwkClass:\ \hmwkTitle}}\\
\normalsize\vspace{0.1in}\small{Due\ on\ \hmwkDueDate}\\
\vspace{0.1in}\large{\textit{\hmwkClassInstructor\ \hmwkClassTime}}
\vspace{3in}
}

\author{\textbf{\hmwkAuthorName}}
\date{} % Insert date here if you want it to appear below your name

%----------------------------------------------------------------------------------------

\begin{document}

%\maketitle

%----------------------------------------------------------------------------------------
%	TABLE OF CONTENTS
%----------------------------------------------------------------------------------------

%\setcounter{tocdepth}{1} % Uncomment this line if you don't want subsections listed in the ToC

%\newpage
%\tableofcontents
%\newpage

%----------------------------------------------------------------------------------------
%	PROBLEM 1
%----------------------------------------------------------------------------------------

% To have just one problem per page, simply put a \clearpage after each problem

\begin{homeworkProblem}
\vspace{10pt}
	
	\textbf{fundamental matrix.m}\newline
	\emph{
	This function computes the fundamental matrix F given the data provided. It should return F as well as the residual res err. The residual is defined as the mean squared distance between the points in the two images and the corresponding epipolar lines. Is this what you are directly optimizing using SVD when solving the homogeneous system? If yes, explain. If no, how does the objective relate to the residual?\newline\newline
	Deliverables: Your answer in your report should include - description of how you calculate F , the definition you are using for the residual, how you are calculating the residual, the fundamental matrix and residuals that you get for the 2 sample inputs, and answers to the questions above. Again, if you like, you may include a code snippet if you think it makes your explanation any easier to understand.
	}
	
	\problemAnswer{
	To calculate F, we first normalize the points by subtracting the mean and scaling such that the variance is approximately equal to 1 (see equations 7.15 and 7.16 in the Szeliski textbook). The final normalized points are centered at (0, 0) and the sum of their squares is equal to 2 times the number of points. Next we create matrix A, where A is an Nx9 matrix with each row equal to:\newline
	\code{
	A = [x1 .* x2, y1 .* x2, x2, x1 .* y2, y1 .* y2, y2, x1, y1, one];\newline
	}
	We can now obtain F from the SVD of A as follows, since we know that the right-most singular value of the decomposition is approximately equal to zero:\newline
	\code{
	\%compute F \newline
	[~,~,V] = svd(A, 0); \newline
	F = reshape(V(:,9), 3, 3)'; \newline
	[FU, FS, FV] = svd(F); \newline
	S = diag([FS(1, 1), FS(2, 2), 0]); \%enforce rank = 2 \newline
	F = FU * S * FV'; \newline
	}
	F is then de-normalized by applying the earlier normalization transformation matrices in reverse.\newline\newline
	The residual is calculated as follows. First epipolar lines are obtained for each of the corresponding points in the image. Next, we obtain the distance between a given point in the image and the estimated epipolar line on which the point should fall. The residual error is calculated as the mean squared distance for all the points. This value corresponds to the minimum of Af (in equation 7 of the assignment), but we also place the additional constraint that F is of rank 2 by changing the values of S before calculating F, so our solution space is the closest rank 2 approximation of the possible solutions to Af=0.\newline\newline
	The residuals we obtained for the two data sets are: \newline
	House (Residual in F = 0.473549 )\newline
	Library (Residual in F = 655.128474)
	}

\end{homeworkProblem}
\clearpage
\begin{homeworkProblem}
\vspace{10pt}
	
	\textbf{find rotation translation.m}\newline
	\emph{
	This function estimates the extrinsic parameters of the second camera. The function should return a cell array R of all the possible rotation matrices and a cell array t with all the possible translation vectors. \newline\newline
	Deliverables: Your answer in your report should include - all possible choices for t and R for the 2 sample inputs.
	}
	
	\problemAnswer{
	House: \newline
	$t = \pm (-0.9994, 0.0202, -0.0277)$\newline\newline
	R = \newline
[   -0.9858,   -0.0688,    0.1532;\newline
    0.0702,   -0.9975,    0.0037;\newline
   -0.1526,   -0.0144,   -0.9882]\newline\newline
[   -0.9947,   -0.0291,    0.0981;\newline
   -0.0301,    0.9995,   -0.0088;\newline
    0.0978,    0.0117,    0.9951]\newline\newline
[    0.9947,    0.0291,   -0.0981;\newline
    0.0301,   -0.9995,    0.0088;\newline
   -0.0978,   -0.0117,   -0.9951]\newline\newline
[    0.9858,    0.0688,   -0.1532;\newline
   -0.0702,    0.9975,   -0.0037;\newline
    0.1526,    0.0144,    0.9882]\newline\newline
    	Library: \newline
	$t = \pm (-0.9984, 0.0050, 0.0560)$\newline\newline
	R = \newline
[    0.9191,    0.0162,   -0.3936;\newline
    0.0164,   -0.9999,   -0.0030;\newline
   -0.3936,   -0.0037,   -0.9193]\newline\newline
[    0.9572,    0.0265,   -0.2884;\newline
   -0.0257,    0.9996,    0.0064;\newline
    0.2884,    0.0013,    0.9575]\newline\newline
[   -0.9572,   -0.0265,    0.2884;\newline
    0.0257,   -0.9996,   -0.0064;\newline
   -0.2884,   -0.0013,   -0.9575]\newline\newline
[   -0.9191,   -0.0162,    0.3936;\newline
   -0.0164,    0.9999,    0.0030;\newline
    0.3936,    0.0037,    0.9193]\newline\newline
	}

\end{homeworkProblem}
\clearpage
\begin{homeworkProblem}
\vspace{10pt}
	
	
	\textbf{find 3d points.m} \newline
	\emph{
	This function reconstructs the 3D point cloud. In particular, it returns a N ? 3 array called points, where N is the number of the corresponding matches. It also returns the reconstruction error rec err, which is defined as the mean distance between the 2D points and the projected 3D points in the two images. Is this reconstruction error what you are directly optimizing when solving the linear system of equations? If yes, explain. If no, how does the objective of your optimization problem relate to this error? \newline\newline
	Deliverables: Your answer in your report should include - exactly how you calculate the reconstruction error, and reconstruction error for the 2 sample inputs, and answers to the questions above.
	}
	
	\problemAnswer{ 
	
	We calculate the reconstruction error by reprojecting the 3D points we obtained via triangulation back into each image plane, and then calculating the mean squared distance between the original image point and the reprojected point. Since this problem is posed as a least squares solution when W=1, we find the optimal solution that satisfies the constraints:\newline\newline
	$x1 = P1 * X$\newline
	$x2 = P2 * X$\newline\newline
	In this case we are directly optimizing the reconstruction error with our method, since a least squares solution by definition minimizes the sum of the squares of the errors.\newline\newline
	The reconstruction errors for the two data sets are: \newline
	House (Reconstruction error = 0.238667),\newline
	Library (Reconstruction error = 0.332657).
			
	}

\end{homeworkProblem}
\clearpage
\begin{homeworkProblem}
\vspace{10pt}
	
	\textbf{plot 3d.m}\newline
	\emph{
	This function plots the 3D points in a 3D plot and displays the camera centers for both cameras. Plotting in 3D can be done using the plot3 command. A plot of the point cloud and the camera centers is enough for the purpose of the assignment. \newline\newline
	Deliverables: Your answer in your report should include - a plot for each of the 2 sample inputs from a reasonable viewpoint.
	}
	
	\problemAnswer{
	Our plots are shown below. Camera 1 is at the origin, and the color indicates order of the input points. ``Height'' corresponds to ``Depth'' as in distance from the camera.\newline\newline
	House		
	\begin{center}
	\includegraphics[width=0.6\columnwidth]{house_result.eps} % Example image
	\end{center} 
	\bigskip
	Library
	\begin{center}
	\includegraphics[width=0.6\columnwidth]{library_result.eps} % Example image
	\end{center}
	}

\end{homeworkProblem}

\end{document}
